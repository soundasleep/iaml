An IAML model instance may contain any number of \modelLink{Sessions}. The \event{onInit} \modelLink{Event Trigger} for a \modelLink{Session} is executed when a \modelLink{Session} is accessed for the first time by a user, in the following way.

Every \textit{visitor} to a web application is assigned a unique identifier in order to promote them to a \textit{user}. If a \modelLink{Session} is accessed by a user and no association exists between the users' unique identifier and the given \modelLink{Session}, then a new relationship is created and stored as part of the \modelLink{Internet Application}. The current user now \textit{owns} that \modelLink{Session}. When this occurs, the \event{onInit} event for that \modelLink{Session} is triggered.

If the current \textit{user} restarts their web browser, the \textit{user} will lose their unique identifier and will be demoted back into a \textit{visitor}. Once a unique identifier has been lost, there is no way to regain access to a particular \modelLink{Session}, and a new \modelLink{Session} relationship will have to be defined. Sessions may also \textit{time out} after a specified period of activity, and this period of activity should be configurable.

The \textit{storage semantics} for a \modelLink{Session} are as follows: For a \modelLink{Value} directly contained within a \modelLink{Session}, any stored value is only accessible to the user that \textit{owns} the containing \modelLink{Session}. Once the \modelLink{Session} ownership has been lost, any stored value is no longer accessible. Similarly, the values stored within a \modelLink{Domain Iterator} (including the current \textit{instance pointer}) are only accessible to the current \textit{user}.
